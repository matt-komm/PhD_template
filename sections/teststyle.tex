\chapter{Introduction of the}

\lipsum[1]

\begin{align}
\sqrt{7\,TeV} & =100\,{GeV}
\end{align}

%TODO
\lipsum[4]

\lipsum[2]

\lipsum[1] Blu blub Blu blubBlu blubBlu blubBlu blub Blu blub Blu blub Blu blubBlu blubBlu blub Blu blub 7\,GeV = 100\,GeV. Blu blubBlu blub Blu blubBlu blub Blu blub Blu blub Blu blubBlu blubBlu blub Blu blub

\begin{align}
\sqrt{7\,TeV}&=100\,{GeV}+\mathrm{GeV}+\sum_{i}\sqrt{\frac{\Big(\nabla\mathcal{A}+\xi{B}-\vec{x}^{2}\Big)_{i}^{\dagger}}{x-y^{3}}} \\
&=100\,{GeV}+0 \\
&=150\,{GeV}+0
\end{align}


This is the theory of \glshere{qft}.

\lipsum[4]


\section{Reconstruction}


\subsection{Reconstruction}

\lipsum[2]

\lipsum[1]


\chapter{Introduction of the CMS x x detector and the $\phi$-system hu hu}

\lipsum[4]

\lipsum[2]

\lipsum[2]

\lipsum[1]

\lipsum[2]

\begin{table}[th]
\caption{This is a table. This is a table. This is a table. This is a table.
This is a table. This is a table. This is a table. This is a table.}


\centering{}%
\begin{tabular}{|c|c|c|c|c|}
\hline 
 &  &  &  & \tabularnewline
\hline 
\hline 
 &  &  &  & \tabularnewline
\hline 
 &  &  &  & \tabularnewline
\hline 
 &  &  &  & \tabularnewline
\hline 
 &  &  &  & \tabularnewline
\hline 
\end{tabular}
\end{table}


\lipsum[1]bla blu blub\footnote{\lipsum[1]}. Bla blu blub.\lipsum[1]

\lipsum[2]

\lipsum[1]bla blu blub\footnote{bla blu blub.}. bla blu blub\footnote{bla blu blub.}. bla blu blub\footnote{bla blu blub.}. Bla blu blub.\lipsum[1]

\lipsum[2]

This is the top quark observation~\cite{topQuarkObservation}. Because \gls{qft} is awesome.


\section{Reconstruction tector and the $\phi$-system hu hu}



\section{Reconstruction}

\lipsum[2]


\subsubsection{test test}

\lipsum[1]

\begin{equation}
\sqrt{7\,TeV}=100\,{\ GeV}
\end{equation}


\lipsum[4]

\lipsum[2]
\begin{figure}[th]
\begin{center}
\missingfigure[figwidth=6cm]{Testing a long text string}
\end{center}

\caption{Test Figure. Test Figure. Test Figure. Test Figure. Test Figure. Test
Figure. Test Figure. Test Figure. Test Figure. Test Figure. Test Figure.
Test Figure. Test Figure. Test Figure. Test Figure. Test Figure. Test
Figure. Test Figure. Test Figure. Test Figure. Test Figure.}
\end{figure}


\lipsum[1]

\begin{equation}
\sqrt{7\,TeV}=100\,{\ GeV}
\end{equation}


\lipsum[4]

\lipsum[2]

\begin{figure}[th]
\begin{center}
\missingfigure[figwidth=6cm]{Testing a long text string}
\end{center}

\caption{Test Figure. Test Figure. Test Figure. Test Figure. Test Figure. Test
Figure. Test Figure. Test Figure. Test Figure. Test Figure. Test Figure.
Test Figure. Test Figure. Test Figure. Test Figure. Test Figure. Test
Figure. Test Figure. Test Figure. Test Figure. Test Figure.}
\end{figure}


\lipsum[4]


\subsection{Reconstruction}

\lipsum[1]

\begin{equation}
\sqrt{7\,TeV}=100\,{\ GeV}
\end{equation}

Have a look at Ref.~\cite{pdg}.

\lipsum[4]

\cleardoublepage
\renewcommand{\chaptername}{Appendix}
\renewcommand\thechapter{\sc\Alph{chapter}}
\setcounter{chapter}{0}
\appendix % do not forget - otherwise toc messed up!
\addcontentsline{toc}{chapter}{\chaptername}

\appendixchapter{Additional Material}

\lipsum[2]

\appendixchapter{Some plots}

\lipsum[2]

\cleardoublepage
%\pagenumbering{Alph}
%\setcounter{page}{1}




\lipsum[4]

\lipsum[2]

\begin{itemize}
\item \lipsum[1]
\item Test item
\item \lipsum[2]
\end{itemize}

\lipsum[4]

\begin{enumerate}
\item \lipsum[1]
\item blublu blu b blue \texttt{Test item} quark quark
\item \lipsum[2]
\end{enumerate}

\lipsum[2]

\begin{figure}[th]
\begin{center}
\missingfigure[figwidth=6cm]{Testing a long text string}
\end{center}

\caption{Test Figure. Test Figure. Test Figure. Test Figure. Test Figure. Test
Figure. Test Figure. Test Figure. Test Figure. Test Figure. Test Figure.
Test Figure. Test Figure. Test Figure. Test Figure. Test Figure. Test
Figure. Test Figure. Test Figure. Test Figure. Test Figure.}
\end{figure}


\lipsum[2]

\lipsum[2]

\begin{figure}[th]
\begin{center}
\missingfigure[figwidth=6cm]{Testing a long text string}
\end{center}

\caption{Test Figure. Test Figure. Test Figure. Test Figure. Test Figure. Test
Figure. Test Figure. Test Figure. Test Figure. Test Figure. Test Figure.
Test Figure. Test Figure. Test Figure. Test Figure. Test Figure. Test
Figure. Test Figure. Test Figure. Test Figure. Test Figure.}
\end{figure}


\lipsum[2]

\lipsum[2]

\begin{figure}[th]
\begin{center}
\missingfigure[figwidth=6cm]{Testing a long text string}
\end{center}

\caption{Test Figure. Test Figure. Test Figure. Test Figure. Test Figure. Test
Figure. Test Figure. Test Figure. Test Figure. Test Figure. Test Figure.
Test Figure. Test Figure. Test Figure. Test Figure. Test Figure. Test
Figure. Test Figure. Test Figure. Test Figure. Test Figure.}
\end{figure}


\lipsum[2]

\begin{figure}[th]
\begin{center}
\missingfigure[figwidth=6cm]{Testing a long text string}
\end{center}

\caption{Test Figure. Test Figure. Test Figure. Test Figure. Test Figure. Test
Figure. Test Figure. Test Figure. Test Figure. Test Figure. Test Figure.
Test Figure. Test Figure. Test Figure. Test Figure. Test Figure. Test
Figure. Test Figure. Test Figure. Test Figure. Test Figure.}
\end{figure}


\lipsum[2]
